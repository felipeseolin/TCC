% CRONOGRAMA

\chapter{CRONOGRAMA}
\label{chap:cronograma}

\par Este capítulo tem como objetivo apresentar um cronograma para o futuro deste projeto. As fases, bem como suas atividades estão listadas logo em seguida e podem ser melhor observadas por meio da \autoref{tab:cronograma} e também através da \autoref{tab:cronograma-fases} as quais fazem uma relação entre atividades, fases e datas.

Vale ressaltar que as tabelas apresentam um mês e logo abaixo o número 1 e 2, o número 1 faz referência à primeira metade do mês e o número 2 à segunda metade\footnote{Exemplo: Dado o mês de novembro, o número 1 faz referência do dia 1 ao 15, já o número 2 faz referência do dia 16 ao 30.}.


\begin{enumerate}
    \item Correções
        \begin{enumerate}
            \item[1.1] Fazer as devidas correções neste documento, conforme detalhes apresentados pela banca e pelo orientador;
            \item[1.2] Verificar e validar correções.
        \end{enumerate}

    \item Preparação
        \begin{enumerate}
            \item[2.1] Definir premissas;
            \item[2.2] Definir módulos;
            \item[2.3] Verificar e validar;
            \item[2.4] Documentar e escrever sobre esta fase.
        \end{enumerate}

    \item Implementação
        \begin{enumerate}
            \item[3.1] Aprofundar conhecimentos em linguagem \textit{python};
            \item[3.2] Estudar novamente a documentação das bibliotecas e demais ferramentas, focando nos métodos que serão utilizados;
            \item[3.3] Iniciar a programação módulo $n$;
            \item[3.4] Testar o módulo $n$;
            \item[3.5] Documentar e escrever sobre esta fase.
        \end{enumerate}
        
    \item Integração
        \begin{enumerate}
            \item[4.1] Integrar os módulos da fase anterior;
            \item[4.2] Testar a integração;
            \item[4.3] Preparar o teste para a validação do algoritmo;
            \item[4.4] Testar o algoritmo;
            \item[4.5] Documentar e escrever sobre esta fase.
        \end{enumerate}
        
    \item Testes e comparações
        \begin{enumerate}
            \item[5.1] Estudar outros algoritmos de descrição de imagens;
            \item[5.2] Preparar testes entre o algoritmo obtido na fase anterior com os presentes na literatura;
            \item[5.3] Realizar testes;
            \item[5.4] Analisar resultados;
            \item[5.5] Verificar e validar;
            \item[5.6] Documentar e escrever sobre esta fase.
        \end{enumerate}
    
    \item Operação e manutenção
        \begin{enumerate}
            \item[6.1] Observar os resultados da etapa anterior e fazer as devidas modificações para a melhor eficiência do algoritmo;
            \item[6.2] Documentar e escrever sobre esta fase.
        \end{enumerate}
        
    \item Documentação
        \begin{enumerate}
            \item[7.1] Finalizar o documento referente ao Trabalho de Conclusão de Curso 2, utilizando da documentação das fases anteriores, além do documento produzido para o Trabalho de Conclusão de Curso 1;
            \item[7.2] Preparar apresentação.
        \end{enumerate}
        
\end{enumerate}

% Please add the following required packages to your document preamble:
% \usepackage[table,xcdraw]{xcolor}
% If you use beamer only pass "xcolor=table" option, i.e. \documentclass[xcolor=table]{beamer}
\begin{table}[H]
\centering
\caption{Cronograma do projeto por atividades.}
\label{tab:cronograma}
\resizebox{\textwidth}{!}{
\begin{tabular}{|l|l|l|l|l|l|l|l|l|l|l|l|l|l|l|l|l|}
\hline
\multicolumn{17}{|c|}{Cronograma}                                                                                                                                                                                                                                                                                                                                                                                                                                                                                                                     \\ \hline
Ano       & \multicolumn{4}{c|}{2019}                                                                                                        & \multicolumn{12}{c|}{2020}                                                                                                                                                                                                                                                                                                                                                                             \\ \hline
Mês       & \multicolumn{2}{c|}{Novembro}                                              & \multicolumn{2}{c|}{Dezembro}                       & \multicolumn{2}{c|}{Janeiro}                        & \multicolumn{2}{c|}{Fevereiro}                      & \multicolumn{2}{c|}{Março}                                                                        & \multicolumn{2}{c|}{Abril}                                                 & \multicolumn{2}{c|}{Maio}                           & \multicolumn{2}{c|}{Junho}                          \\ \hline
Atividade & \multicolumn{1}{c|}{1}   & \multicolumn{1}{c|}{2}                          & \multicolumn{1}{c|}{1}   & \multicolumn{1}{c|}{2}   & \multicolumn{1}{c|}{1}   & \multicolumn{1}{c|}{2}   & \multicolumn{1}{c|}{1}   & \multicolumn{1}{c|}{2}   & \multicolumn{1}{c|}{1}                          & \multicolumn{1}{c|}{2}                          & \multicolumn{1}{c|}{1}                          & \multicolumn{1}{c|}{2}   & \multicolumn{1}{c|}{1}   & \multicolumn{1}{c|}{2}   & \multicolumn{1}{c|}{1}   & \multicolumn{1}{c|}{2}   \\ \hline
1.1       & \cellcolor[HTML]{000000} &                                                 &                          &                          &                          &                          &                          &                          &                                                 &                                                 &                                                 &                          &                          &                          &                          &                          \\ \hline
1.2       & \cellcolor[HTML]{000000} &                                                 &                          &                          &                          &                          &                          &                          &                                                 &                                                 &                                                 &                          &                          &                          &                          &                          \\ \hline
2.1       &                          & \cellcolor[HTML]{000000}{\color[HTML]{000000} } &                          &                          &                          &                          &                          &                          &                                                 &                                                 &                                                 &                          &                          &                          &                          &                          \\ \hline
2.2       &                          & \cellcolor[HTML]{000000}                        &                          &                          &                          &                          &                          &                          &                                                 &                                                 &                                                 &                          &                          &                          &                          &                          \\ \hline
2.3       &                          & \cellcolor[HTML]{000000}                        &                          &                          &                          &                          &                          &                          &                                                 &                                                 &                                                 &                          &                          &                          &                          &                          \\ \hline
2.4       &                          & \cellcolor[HTML]{000000}                        &                          &                          &                          &                          &                          &                          &                                                 &                                                 &                                                 &                          &                          &                          &                          &                          \\ \hline
3.1       &                          &                                                 & \cellcolor[HTML]{000000} &                          &                          &                          &                          &                          &                                                 &                                                 &                                                 &                          &                          &                          &                          &                          \\ \hline
3.2       &                          &                                                 & \cellcolor[HTML]{000000} &                          &                          &                          &                          &                          &                                                 &                                                 &                                                 &                          &                          &                          &                          &                          \\ \hline
3.3       &                          &                                                 & \cellcolor[HTML]{000000} & \cellcolor[HTML]{000000} & \cellcolor[HTML]{000000} & \cellcolor[HTML]{000000} &                          &                          &                                                 &                                                 &                                                 &                          &                          &                          &                          &                          \\ \hline
3.4       &                          &                                                 & \cellcolor[HTML]{000000} & \cellcolor[HTML]{000000} & \cellcolor[HTML]{000000} & \cellcolor[HTML]{000000} &                          &                          &                                                 &                                                 &                                                 &                          &                          &                          &                          &                          \\ \hline
3.5       &                          &                                                 &                          &                          &                          & \cellcolor[HTML]{000000} &                          &                          &                                                 &                                                 &                                                 &                          &                          &                          &                          &                          \\ \hline
4.1       &                          &                                                 &                          &                          &                          &                          & \cellcolor[HTML]{000000} &                          &                                                 &                                                 &                                                 &                          &                          &                          &                          &                          \\ \hline
4.2       &                          &                                                 &                          &                          &                          &                          & \cellcolor[HTML]{000000} &                          &                                                 &                                                 &                                                 &                          &                          &                          &                          &                          \\ \hline
4.3       &                          &                                                 &                          &                          &                          &                          & \cellcolor[HTML]{000000} &                          &                                                 &                                                 &                                                 &                          &                          &                          &                          &                          \\ \hline
4.4       &                          &                                                 &                          &                          &                          &                          & \cellcolor[HTML]{000000} &                          &                                                 &                                                 &                                                 &                          &                          &                          &                          &                          \\ \hline
4.5       &                          &                                                 &                          &                          &                          &                          & \cellcolor[HTML]{000000} &                          &                                                 &                                                 &                                                 &                          &                          &                          &                          &                          \\ \hline
5.1       &                          &                                                 &                          &                          &                          &                          &                          & \cellcolor[HTML]{000000} &                                                 &                                                 &                                                 &                          &                          &                          &                          &                          \\ \hline
5.2       &                          &                                                 &                          &                          &                          &                          &                          & \cellcolor[HTML]{000000} &                                                 &                                                 &                                                 &                          &                          &                          &                          &                          \\ \hline
5.3       &                          &                                                 &                          &                          &                          &                          &                          &                          & \cellcolor[HTML]{000000}{\color[HTML]{000000} } &                                                 &                                                 &                          &                          &                          &                          &                          \\ \hline
5.4       &                          &                                                 &                          &                          &                          &                          &                          &                          & \cellcolor[HTML]{000000}{\color[HTML]{000000} } &                                                 &                                                 &                          &                          &                          &                          &                          \\ \hline
5.5       &                          &                                                 &                          &                          &                          &                          &                          &                          & \cellcolor[HTML]{000000}{\color[HTML]{000000} } &                                                 &                                                 &                          &                          &                          &                          &                          \\ \hline
5.6       &                          &                                                 &                          &                          &                          &                          &                          &                          & \cellcolor[HTML]{000000}{\color[HTML]{000000} } &                                                 &                                                 &                          &                          &                          &                          &                          \\ \hline
6.1       &                          &                                                 &                          &                          &                          &                          &                          &                          &                                                 & \cellcolor[HTML]{000000}{\color[HTML]{000000} } & \cellcolor[HTML]{000000}{\color[HTML]{000000} } & \cellcolor[HTML]{000000} & \cellcolor[HTML]{000000} &                          &                          &                          \\ \hline
6.2       &                          &                                                 &                          &                          &                          &                          &                          &                          &                                                 & \cellcolor[HTML]{000000}{\color[HTML]{000000} } & \cellcolor[HTML]{000000}{\color[HTML]{000000} } & \cellcolor[HTML]{000000} & \cellcolor[HTML]{000000} &                          &                          &                          \\ \hline
7.1       &                          &                                                 &                          &                          &                          &                          &                          &                          &                                                 &                                                 &                                                 & \cellcolor[HTML]{000000} & \cellcolor[HTML]{000000} & \cellcolor[HTML]{000000} & \cellcolor[HTML]{000000} &                          \\ \hline
7.2       &                          &                                                 &                          &                          &                          &                          &                          &                          &                                                 &                                                 &                                                 &                          &                          & \cellcolor[HTML]{000000} & \cellcolor[HTML]{000000} & \cellcolor[HTML]{000000} \\ \hline
\end{tabular}
}
\end{table}

Uma visão mais resumida e menos detalhada pode ser vista na \autoref{tab:cronograma-fases}, estando esta organizada através das fases do projeto, definidas no início deste \autoref{chap:cronograma}.

% Please add the following required packages to your document preamble:
% \usepackage[table,xcdraw]{xcolor}
% If you use beamer only pass "xcolor=table" option, i.e. \documentclass[xcolor=table]{beamer}
\begin{table}[H]
\centering
\caption{Cronograma do projeto por fases.}
\label{tab:cronograma-fases}
\resizebox{\textwidth}{!}{
\begin{tabular}{|l|l|l|l|l|l|l|l|l|l|l|l|l|l|l|l|l|}
\hline
\multicolumn{17}{|c|}{Cronograma}                                                                                                                                                                                                                                                                                                                                                                                                                    \\ \hline
Ano  & \multicolumn{4}{c|}{2019}                                                                                 & \multicolumn{12}{c|}{2020}                                                                                                                                                                                                                                                                                                        \\ \hline
Mês  & \multicolumn{2}{c|}{Novembro}                       & \multicolumn{2}{c|}{Dezembro}                       & \multicolumn{2}{c|}{Janeiro}                        & \multicolumn{2}{c|}{Fevereiro}                      & \multicolumn{2}{c|}{Março}                          & \multicolumn{2}{c|}{Abril}                          & \multicolumn{2}{c|}{Maio}                           & \multicolumn{2}{c|}{Junho}                          \\ \hline
Fase & \multicolumn{1}{c|}{1}   & \multicolumn{1}{c|}{2}   & \multicolumn{1}{c|}{1}   & \multicolumn{1}{c|}{2}   & \multicolumn{1}{c|}{1}   & \multicolumn{1}{c|}{2}   & \multicolumn{1}{c|}{1}   & \multicolumn{1}{c|}{2}   & \multicolumn{1}{c|}{1}   & \multicolumn{1}{c|}{2}   & \multicolumn{1}{c|}{1}   & \multicolumn{1}{c|}{2}   & \multicolumn{1}{c|}{1}   & \multicolumn{1}{c|}{2}   & \multicolumn{1}{c|}{1}   & \multicolumn{1}{c|}{2}   \\ \hline
1    & \cellcolor[HTML]{000000} &                          &                          &                          &                          &                          &                          &                          &                          &                          &                          &                          &                          &                          &                          &                          \\ \hline
2    &                          & \cellcolor[HTML]{000000} &                          &                          &                          &                          &                          &                          &                          &                          &                          &                          &                          &                          &                          &                          \\ \hline
3    &                          &                          & \cellcolor[HTML]{000000} & \cellcolor[HTML]{000000} & \cellcolor[HTML]{000000} & \cellcolor[HTML]{000000} &                          &                          &                          &                          &                          &                          &                          &                          &                          &                          \\ \hline
4    &                          &                          &                          &                          &                          &                          & \cellcolor[HTML]{000000} &                          &                          &                          &                          &                          &                          &                          &                          &                          \\ \hline
5    &                          &                          &                          &                          &                          &                          &                          & \cellcolor[HTML]{000000} & \cellcolor[HTML]{000000} &                          &                          &                          &                          &                          &                          &                          \\ \hline
6    &                          &                          &                          &                          &                          &                          &                          &                          &                          & \cellcolor[HTML]{000000} & \cellcolor[HTML]{000000} & \cellcolor[HTML]{000000} & \cellcolor[HTML]{000000} &                          &                          &                          \\ \hline
7    &                          &                          &                          &                          &                          &                          &                          &                          &                          &                          &                          & \cellcolor[HTML]{000000} & \cellcolor[HTML]{000000} & \cellcolor[HTML]{000000} & \cellcolor[HTML]{000000} & \cellcolor[HTML]{000000} \\ \hline
\end{tabular}
}
\end{table}