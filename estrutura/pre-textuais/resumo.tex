% RESUMO--------------------------------------------------------------------------------

\begin{resumo}[RESUMO]
\begin{SingleSpacing}

% Não altere esta seção do texto--------------------------------------------------------
\imprimirautorcitacao. \imprimirtitulo. \imprimirdata. \pageref {LastPage} f. \imprimirprojeto\ – \imprimirprograma, \imprimirinstituicao. \imprimirlocal, \imprimirdata.\\
%---------------------------------------------------------------------------------------

O mundo está repleto de imagens, as quais possuem uma vasta quantidade de dados que podem ser processados de formas distintas. No contexto de imagens digitais, o seu processamento pode auxiliar diversas áreas como saúde, astronomia, agroindústria, entre outras. A descrição de uma imagem é uma etapa importante dentro do processamento digital de imagens, pois é responsável por identificar texturas, cores ou formas. Para isto, há uma tendência de estudos que buscam diferentes abordagens para algoritmos que descrevam objetos dentro de uma imagem da melhor forma possível. Neste sentido, este trabalho apresenta uma nova abordagem para tal etapa de processamento de imagens digitais, baseada em estudo de grafos, mais especificamente em árvores geradoras mínimas. Assim, obtendo como resultado a análise, discussão e validação deste novo algoritmo extrator. O algoritmo desenvolvido liga todos os \textit{pixels} de uma imagem, de forma a caracterizar vizinhança oito, em seguida pondera todas as arestas a partir do módulo da diferença entre os \textit{pixels}, encontra a árvore geradora mínima e a partir dos valores das arestas calcula algumas medidas de forma a montar um vetor de características. Por fim, pode-se afirmar que ao comparar os resultados do algoritmo proposto com outros já consolidados na literatura e submeter a testes em diferentes cenários, o algoritmo se mostrou eficaz na maioria das vezes, apresentando acurácias próximas aos comparados e até mesmo melhores em determinadas situações.
\\

\textbf{Palavras-chave}: Imagem.  Textura. Processamento Digital de Imagens. Descrição de Imagens. Árvore Geradora Mínima.

\end{SingleSpacing}
\end{resumo}

% OBSERVAÇÕES---------------------------------------------------------------------------
% Altere o texto inserindo o Resumo do seu trabalho.
% Escolha de 3 a 5 palavras ou termos que descrevam bem o seu trabalho 
