% ABSTRACT--------------------------------------------------------------------------------

\begin{resumo}[ABSTRACT]
\begin{SingleSpacing}

% Não altere esta seção do texto--------------------------------------------------------
\imprimirautorcitacao. \imprimirtitleabstract. \imprimirdata. \pageref {LastPage} f. \imprimirprojeto\ – \imprimirprograma, \imprimirinstituicao. \imprimirlocal, \imprimirdata.\\
%---------------------------------------------------------------------------------------

The world is full of images, which contain a vast amount of data that can be processed in different ways. In the context of digital images, its processing can help different areas such as health, astronomy, agribusiness, among others. The description of an image is an important step in digital image processing, as it is responsible for identifying textures, colors or shapes. For this, there is a trend of studies that seek different approaches to algorithms that describe objects within an image in the best possible way. In this sense, this work presents a new approach for this stage of digital image processing, based on the study of graphs, more specifically on minimal spanning trees. Thus, obtaining as a result the analysis, discussion and validation of this new extractor algorithm. The algorithm connects all the pixels of an image, in order to characterize neighborhood eight, then it weights all the edges from the modulus of the difference between the pixels, finds the minimum spanning tree and from the edge values ​​it calculates some shape measures to assemble a vector of features. Finally, it can be said that when comparing the results of the proposed algorithm with others already consolidated in the literature and submitting to tests in different scenarios, the algorithm proved to be effective most of the time, presenting accuracy close to those compared and even better in certain situations.
\\

\textbf{Keywords}: Image. Texture. Digital Image Processing. Images Description. Minimum Spanning Tree.

\end{SingleSpacing}
\end{resumo}

% OBSERVAÇÕES---------------------------------------------------------------------------
% Altere o texto inserindo o Abstract do seu trabalho.
% Escolha de 3 a 5 palavras ou termos que descrevam bem o seu trabalho 
