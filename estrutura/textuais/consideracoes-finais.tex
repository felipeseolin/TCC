% CONCLUSÃO--------------------------------------------------------------------

\chapter{CONSIDERAÇÕES FINAIS}
\label{chap:consideracoesFinais}

\par Por tanto, a proposta deste trabalho é definir e especificar um projeto que procura desenvolver uma nova maneira, de descrever texturas em uma imagem, por meio da utilização de grafos e também de árvores geradoras mínimas.
\par Como observado, a descrição é uma parte importante do processamento digital de imagens, pois é capaz de identificar padrões em um objeto selecionado pela etapa de segmentação.
\par Sendo assim, é esperado que com a conclusão do algoritmo, este seja capaz de descrever texturas em imagens digitais de forma a ser mais eficaz e/ou eficiente quando comparado a outros algoritmos já presentes na literatura. Além disso, espera-se que com este documento, devidamente atualizado após a conclusão, possa ser utilizado para futuras melhorias no algoritmo e também em novas abordagens.

\section{TRABALHOS FUTUROS}
\label{sec:trabalhos-futuros}

\par É de conhecimento que para este estudo ser bem-sucedido é necessário seguir a metodologia, descrita na \autoref{sec:metodo}, em conjunto com o cronograma apresentado no \autoref{chap:cronograma}.
\par Dessa forma, a partir do mês de novembro de 2019 será dada continuidade nos trabalhos descritos neste documento, bem como anteriormente apresentados, e serão devidamente finalizados em junho de 2020.
